
{\actuality} Системы работающие с сверхширокополосными (СШП) сигналами в последние годы вызывают все больший интерес и получили большое распространение за пределами продукции военного и специального назначения. Это вызвано развитием интегральной электроники и постепенным освоением нелицензируемых диапазонов \numrange[]{22}{27}, \numrange[]{63}{64} и \numrange[]{76}{81}~ГГц для реализации автомобильных радаров ближнего и дальнего действия. Помимо радаров помощи водителю (пилота) системы с СШП находят применение в системах контроля состояния водителя и пассажиров, за счет дополнительной обработки и выявления периодичности радиолокационной картины, определяют наличие и характер дыхания, а также частоту сердечных сокращений. Интерес к системам связи работающих на базе сверхширокополосных сигналов возрастает на фоне ограниченности частотных ресурсов сегодняшних беспроводных сетей. Системы с использованием сверхширокополосных сигналов позволяют обеспечивать высокую скорость передачи данных соседствуя с существующими системами связи, за счет расширенного спектра сигнала и малой излучаемой мощности. 

Быстродействие обработки сигналов главного тракта приема сверхширокополосных сигналов является критичной характеристикой всего приемного устройства, так как от него зависит общая пригодность устройства к защите пассажиров, водителя, пилота или беспилотного аппарата и т.п. от внешних угроз. Также работа сверхширокополосного преимника сопровождается сложной электромагнитной обстановкой, например, большое разнообразие радаров и видов сигналов которые они используют, передатчики установленные на той же платформе что и сверхширокополосный приемник, которые могут полностью подавлять прием других сигналов.

Решением задач разработки сверхширокополосных приемных трактов и систем для задач радиолокации, радиопеленгации и распознования образов, занимается ряд отечественных и зарубежных компаний, таких как  <<Northrop Grumman Corp.>>, <<BAE Systems>>, <<Lockheed Martin>> и <<АО ЦКБА>>, а также такие исследователи как James~B.~Tsui, Schmidt~O.~Ralph, Аткишкин~С., Сапожников~Д., Подстригаев~А.~С., Василенко~В.~Э. и др.

Важность исследований, связанных с разработкой интегрированных в системы-на-кристалле (СнК) сверхширокополосных приемных трактов мгновенного измерения частоты (МИЧ), подвтерждается необходимостью большей микроминиатюризации существующих устройств с использованием преимуществ высокоинтегрированных систем, т. е. возможность выполнения сложной цифровой обработки сигналов (ЦОС) в непосредственной близости от радиоприемного тракта, что уменьшает количество высокоскоростных линий передачи данных в итоговом устройстве и общую сложность системы. Помимо вышеперечисленного, интеграция системы приема сверхширокополосных сигналов в СнК позволяет снизить энергопотребление, а также ускорить процесс разработки и облегчить процесс обеспечения электромагнитной совместимости внутри устройства между блоками.

\ifsynopsis
Этот абзац появляется только в~автореферате.
Для формирования блоков, которые будут обрабатываться только в~автореферате,
заведена проверка условия \verb!\!\verb!ifsynopsis!.
Значение условия задаётся в~основном файле документа (\verb!synopsis.tex! для
автореферата).
\else
Этот абзац появляется только в~диссертации.
Через проверку условия \verb!\!\verb!ifsynopsis!, задаваемого в~основном файле
документа (\verb!dissertation.tex! для диссертации), можно сделать новую
команду, обеспечивающую появление цитаты в~диссертации, но~не~в~автореферате.
\fi

{\aim} данной работы является исследование и разработка ряда приемных трактов сверхширокополосных сигналов в диапазоне \numrange[]{1}{40}~ГГц, для задач пассивной радиолокации, радиоэлектронной поддержки летательных и транспортных средств, в реализации системы на кристалле или системы в корпусе. Целью данной работы является формирование ряда методов проектирования многоканальных сверхширокополосных трактов приема сверхкоротких импульсов в интегральном исполнении, позволяющих снизить затраты на проектирование, отладку и производство.

Для~достижения поставленной цели необходимо было решить следующие {\tasks}:
\begin{enumerate}[beginpenalty=10000] % https://tex.stackexchange.com/a/476052/104425
  \item Исследовать современные структуры приемных трактов, повышения быстродействия в реализации систем на кристалле или систем в корпусе;
  \item Исследовать методы проектирования гибридных приемников измерения частоты в реализациях систем на кристалле с малой потребляемой мощностью и минимизацией по площади;
  \item Разработать математическую модель тракта приема СШП сигналов с учетом паразитных эффектов оказывающих влияние в интегральном исполнении.
\end{enumerate}

{\novelty}
\begin{enumerate}[beginpenalty=10000] % https://tex.stackexchange.com/a/476052/104425
  \item Впервые \ldots
  \item Впервые \ldots
  \item Было выполнено оригинальное исследование \ldots
\end{enumerate}

{\influence} \ldots

{\methods}
При решении поставленных задач применяются методы теории линейных и нелинейных электрических цепей, математического анализа, вычислительной математики, применяются специализированные системы моделирования, автоматизированного проектирования, а также математические пакеты с функциями численных методов решения уравнений и оптимизации.

{\defpositions}
\begin{enumerate}[beginpenalty=10000] % https://tex.stackexchange.com/a/476052/104425
  \item Первое положение
  \item Второе положение
  \item Третье положение
  \item Четвертое положение
\end{enumerate}
В папке Documents можно ознакомиться с решением совета из Томского~ГУ
(в~файле \verb+Def_positions.pdf+), где обоснованно даются рекомендации
по~формулировкам защищаемых положений.

{\reliability} полученных результатов обеспечивается использованием математического аппарата и современных пакетов компьютерного моделирования Cadence, ADS Systems, Empire с использованием адекватных моделей полупроводниковых элементов представляемых производителями, а также результатами измерений опытных образцов с использованием установки зондовых измерений MPI TS200. Экспериментальные исследования продемонстрировали соответствие разработанных моделей действительности. Результаты находятся в соответствии с результатами, полученными другими авторами. Также достоверность подтверждается выступлениями на международных конференциях IEEE и наличием актов внедрения.

{\probation}
Основные результаты работы докладывались на:
международной конференции Динамика механизмов и машин, всероссийской научно-практической конференции СВЧ-2022, всероссийской конференции Микроэлектроника-2022 (Школа Молодых Ученых 2022 ШМУ-2022), международная конференция IEEE EDM2023, перечисление основных конференций, симпозиумов и т.\:п.

{\contribution} Все изложенные в диссертации исследования выполнены непосредственно ее автором. Из публикаций, написанных в соавторстве, в диссертацию включен лишь тот материал, который принадлежит непосредственно соискателю. Все результаты, сформулированные в положениях, выносимых на защиту и составляющие научную новизну работы, получены лично автором, в том числе выбор методик исследований, разработку алгоритмов программных решений, проведение измерений полученных образцов и обработку экспериментальных результатов.

\ifnumequal{\value{bibliosel}}{0}
{%%% Встроенная реализация с загрузкой файла через движок bibtex8. (При желании, внутри можно использовать обычные ссылки, наподобие `\cite{vakbib1,vakbib2}`).
    {\publications} Основные результаты по теме диссертации изложены
    в~XX~печатных изданиях,
    X из которых изданы в журналах, рекомендованных ВАК,
    X "--- в тезисах докладов.
}%
{%%% Реализация пакетом biblatex через движок biber
    \begin{refsection}[bl-author, bl-registered]
        % Это refsection=1.
        % Процитированные здесь работы:
        %  * подсчитываются, для автоматического составления фразы "Основные результаты ..."
        %  * попадают в авторскую библиографию, при usefootcite==0 и стиле `\insertbiblioauthor` или `\insertbiblioauthorgrouped`
        %  * нумеруются там в зависимости от порядка команд `\printbibliography` в этом разделе.
        %  * при использовании `\insertbiblioauthorgrouped`, порядок команд `\printbibliography` в нём должен быть тем же (см. biblio/biblatex.tex)
        %
        % Невидимый библиографический список для подсчёта количества публикаций:
        \printbibliography[heading=nobibheading, section=1, env=countauthorvak,          keyword=biblioauthorvak]%
        \printbibliography[heading=nobibheading, section=1, env=countauthorwos,          keyword=biblioauthorwos]%
        \printbibliography[heading=nobibheading, section=1, env=countauthorscopus,       keyword=biblioauthorscopus]%
        \printbibliography[heading=nobibheading, section=1, env=countauthorconf,         keyword=biblioauthorconf]%
        \printbibliography[heading=nobibheading, section=1, env=countauthorother,        keyword=biblioauthorother]%
        \printbibliography[heading=nobibheading, section=1, env=countregistered,         keyword=biblioregistered]%
        \printbibliography[heading=nobibheading, section=1, env=countauthorpatent,       keyword=biblioauthorpatent]%
        \printbibliography[heading=nobibheading, section=1, env=countauthorprogram,      keyword=biblioauthorprogram]%
        \printbibliography[heading=nobibheading, section=1, env=countauthor,             keyword=biblioauthor]%
        \printbibliography[heading=nobibheading, section=1, env=countauthorvakscopuswos, filter=vakscopuswos]%
        \printbibliography[heading=nobibheading, section=1, env=countauthorscopuswos,    filter=scopuswos]%
        %
        \nocite{*}%
        %
        {\publications} Основные результаты по теме диссертации изложены в~\arabic{citeauthor}~печатных изданиях,
        \arabic{citeauthorvak} из которых изданы в журналах, рекомендованных ВАК\sloppy%
        \ifnum \value{citeauthorscopuswos}>0%
            , \arabic{citeauthorscopuswos} "--- в~периодических научных журналах, индексируемых Web of~Science и Scopus\sloppy%
        \fi%
        \ifnum \value{citeauthorconf}>0%
            , \arabic{citeauthorconf} "--- в~тезисах докладов.
        \else%
            .
        \fi%
        \ifnum \value{citeregistered}=1%
            \ifnum \value{citeauthorpatent}=1%
                Зарегистрирован \arabic{citeauthorpatent} патент.
            \fi%
            \ifnum \value{citeauthorprogram}=1%
                Зарегистрирована \arabic{citeauthorprogram} программа для ЭВМ.
            \fi%
        \fi%
        \ifnum \value{citeregistered}>1%
            Зарегистрированы\ %
            \ifnum \value{citeauthorpatent}>0%
            \formbytotal{citeauthorpatent}{патент}{}{а}{}\sloppy%
            \ifnum \value{citeauthorprogram}=0 . \else \ и~\fi%
            \fi%
            \ifnum \value{citeauthorprogram}>0%
            \formbytotal{citeauthorprogram}{программ}{а}{ы}{} для ЭВМ.
            \fi%
        \fi%
        % К публикациям, в которых излагаются основные научные результаты диссертации на соискание учёной
        % степени, в рецензируемых изданиях приравниваются патенты на изобретения, патенты (свидетельства) на
        % полезную модель, патенты на промышленный образец, патенты на селекционные достижения, свидетельства
        % на программу для электронных вычислительных машин, базу данных, топологию интегральных микросхем,
        % зарегистрированные в установленном порядке.(в ред. Постановления Правительства РФ от 21.04.2016 N 335)
    \end{refsection}%
    \begin{refsection}[bl-author, bl-registered]
        % Это refsection=2.
        % Процитированные здесь работы:
        %  * попадают в авторскую библиографию, при usefootcite==0 и стиле `\insertbiblioauthorimportant`.
        %  * ни на что не влияют в противном случае
        \nocite{vakbib2}%vak
        \nocite{patbib1}%patent
        \nocite{progbib1}%program
        \nocite{bib1}%other
        \nocite{confbib1}%conf
    \end{refsection}%
        %
        % Всё, что вне этих двух refsection, это refsection=0,
        %  * для диссертации - это нормальные ссылки, попадающие в обычную библиографию
        %  * для автореферата:
        %     * при usefootcite==0, ссылка корректно сработает только для источника из `external.bib`. Для своих работ --- напечатает "[0]" (и даже Warning не вылезет).
        %     * при usefootcite==1, ссылка сработает нормально. В авторской библиографии будут только процитированные в refsection=0 работы.
}
