\documentclass[a4paper,14pt,oneside,openany]{memoir}

\pagestyle{empty}

\usepackage[a4paper, top=2cm, bottom=2cm,
           left=2.5cm, right=1cm, nofoot, nohead,
           nomarginpar, heightrounded, showframe]{geometry}
\usepackage{polyglossia}
\setmainlanguage{russian}
\setotherlanguage{english}
\setmainfont{LiberationSerif}
\newfontfamily\cyrillicfont{LiberationSerif}
\setsansfont{LiberationSans}
\newfontfamily\cyrillicfontsf{LiberationSans}

\usepackage{tikz}
\usetikzlibrary{
	arrows.meta,
	automata, %  is used for drawing "finite state automata and Turing Machines". To draw these graphs, each node, its name and relative position is defined, as well as the types of path between each
	backgrounds, % defines background for pictures
	calc, % make complex coordinate calculations
	calendar, % display calendars
	chains, % align nodes o chains
	decorations, % decorate paths
	decorations.pathmorphing,
	er, % each node is defined, as is each edge between each node, as well as any attributes
	graphs, % draw graphs
	graphdrawing,
	intersections, % calculate intersections of paths
	mindmap,
	matrix, % display matrices
	math,
	folding, % paper folding library
	patterns, % defines patterns for filling areas
	petri, % draw Petri nets
	plotmarks, % additional styles for plots
	shapes, % define shapes other than rectangle, circle and coordinate
	trees, % Each point on the tree is defined as a node, with children, and each child can have its own children
}
\usegdlibrary{trees}

\usepackage{pgfplots}
\usepgfplotslibrary{groupplots}
\pgfplotsset{compat=newest}
\pgfkeys{/pgf/number format/.cd,use comma,1000 sep={}}

\usepackage{siunitx}
\usepackage[european,cuteinductors]{circuitikz}

\begin{document}
\fbox{\begin{circuitikz}[american, scale=1, transform shape]
	\def\killdepth#1{{\raisebox{0pt}[\height][0pt]{#1}}} \path (0,0) -- (2,0); % bounding box
	\ctikzset{transistors/arrow pos=end}
	\ctikzset{transistors/scale=1}
	\ctikzset{resistors/scale=0.8}
	\ctikzset{capacitors/scale=0.8}
	
	\draw (0, 0) node[npn, anchor=B](Q1){Q1};
	\draw (Q1.B) [short, -] to ++(-1,0) [short, *-] to [C] ++(-2,0) [short, -o, l=RF Input] to ++(-1,0);
	\draw (Q1.B) [short, -] ++(-1,0) to ++(0, 1) to [R=\(R_{1}\)] ++(0, 2) node[vcc](VCC){};
	\draw (Q1.B) [short, -] ++(-1,0) to ++(0, -1) to [R=\(R_{2}\)] ++(0,-2) node[ground](GND){};
	\draw (Q1.C) [short, -] to ++(0, 2) coordinate(tmp) to[R, l_=\(R_3\)] ++(0, 2) node[vcc](VCC){\(V_{CC}=2.5 V\)};
	\draw (Q1.E) [short, -] to [R, l=\(R_{E}\)] +(0,-2) node[ground](GND){};

	\draw (tmp) node[circ](){} [short, -] to ++(1,0) coordinate(tmp) to [R] ++(2,0) to [short, -] ++(2,0) node[circ](){} to ++(1,0) node[circ](){} to [R] ++(0,-2) node[ground](){};
	\draw (tmp) node[circ](){} [short, -] to ++(0,-1) to [R] ++(2,0) to [C] ++(2,0) to ++(0,1);
	
	%\draw (Q1.C) [short, *-] to [C] ++(2,0) coordinate(tmp);
	%\draw (tmp) node[npn, anchor=B](Q2){Q2};
	%\draw (Q2.E) to[R] ++(0,-2) node[ground](GND){};
	%\draw (Q2.E) [short, *-] to ++(1,0) coordinate(tmp);
	%\draw (tmp) to [C] ++(0,-2) node[ground](GND){};
	%\draw (Q2.C) node[npn, anchor=E](Q3){Q3};
	%\draw (Q3.C) to[R] ++(0,2) node[vcc](VCC){\(V_{CC}\)};
	%\draw (Q3.B) [short, -] to (Q3.B |- VCC) [short, -*] to (VCC);
	
	%\draw (Q3.C) [short, *-] to [C] ++(2,0) coordinate(tmp);
	%\draw (tmp) node[npn, anchor=B](Q4){Q4};
	%\draw (Q4.C) to [R] ++(0,2) node[vcc](VCC){\(V_{CC}\)};
	%\draw (Q4.E) [short, -] to ++(0, -1) to ++(2,0) coordinate(tmp) to [R] ++(0,-2) node[ground](GND){};
	%\draw (tmp) [short, *-] to ++(2, 0) coordinate(tmp);
	%\draw (tmp) [short, -] to ++(0, 1) coordinate(tmp);
	%\draw (tmp) node[npn, anchor=E, xscale=-1](Q5){\scalebox{-1}[1]{Q5}};
	%\draw (Q5.C) to [R] ++(0,2) node[vcc](VCC){\(V_{CC}\)};
	%\draw (Q5.B) [short, -] to [C] ++(0,-2) node[ground](GND){};
	
	%\draw (Q4.C) [short, *-o, l=\(v_{out-}\)] to ++(1.2,0);
	%\draw (Q5.C) [short, *-o, l_=\(v_{out+}\)] to ++(-1.2,0);
	
	%COOOOOORDS
	%\path (tmp) \coord(tmp);
\end{circuitikz}}
\end{document}
