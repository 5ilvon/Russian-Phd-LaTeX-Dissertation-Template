\chapter{Схемотехнический расчет СФ-блоков}

\section{Устройство выборки-хранения}
Определим мощность шумов квантования идеального АЦП с разрядностью \(N\) и размахом полной шкалы \(V_{FS}\)
\[ {P_{q}}^2 = \frac{\Delta^2}{12}\]
если \(\Delta = \dfrac{V_{FS}}{2^N}\), тогда \( {P_q}^2 = 223.52~\mathrm{pW} \).

\section*{Технические требования к аналого-цифровому преобразователю}
Характеристики приведены в Табл.~\ref{tab:Parameters}.
\begin{table}[h]
	\caption[Характеристики разрабатываемого АЦП]{Характеристики разрабатываемого АЦП}
	\label{tab:Parameters}
	\centering
	\begin{tabular}{rrr}
		\toprule
		\textbf{Параметр}      & \textbf{Значение} & \textbf{Единица измерения}\\
		\midrule
		\(N\)                 &   14      & Бит\\
		\(F_{clk}\)           &   250     & МГц\\
		\(V_{FS}\)            &   1.2     & В\\
		\(V_{CM_{in}}\)                   &   600     & мВ\\
		\(FPBW\) (Full Power Bandwidth)   & 750 & МГц\\
		\bottomrule
	\end{tabular}
\end{table}

УВХ построено на базе МОП ключа Q1 с цепью смещения напряжения затвора \(V_{gs1}\), позволяющей снизить нелинейность ключа, возникающую из-за неполного <<закрытия>> или <<открытия>> канала.

\begin{figure}[ht]
	\centering
	\begin{circuitikz}[american, scale=1, transform shape]
	\ctikzset{tripoles/mos style/arrows}
	\ctikzset{bipoles/thickness=2}
	\ctikzset{tripoles/pmos style/emptycircle}
	\def\killdepth#1{{\raisebox{0pt}[\height][0pt]{#1}}} \path (0,0) -- (2,0); % bounding box
	\draw (0,0) node[nmos,rotate=-90](Q1){\rotatebox{90}{Q1}};
	\draw (Q1.D) to [short, l=Out, -] ++(2,0) to[C, l2^=$C_h$ and 4 pF]
	++(0,-1) node[ground](GND){};    ++(0,-1) node[ground](GND){};
	
	
	\draw ++(-3,0) coordinate(in) node[left](vi1){$v_i=v_1$} [short, l=Input, o-] to (Q1.S);
	\draw (in) ++(2,0) [short, *-] to ++(0,2) node[nmos, rotate=-90, anchor=D](Q2){\rotatebox{90}{Q2}};
	\draw (Q2.G) [short, -] to (Q2.G -| Q1.G) [short,-] to (Q1.G);
	\draw (Q2.S) node[nmos, rotate=-90, anchor=D](Q3){\rotatebox{90}{Q3}};
	\draw (Q3.D) [short, *-] to [C] ++(0, 4) coordinate(temp);
	\draw (temp) node[pmos, bulk, rotate=90, anchor=D, xscale=-1](Q4){\scalebox{1}[-1]{\rotatebox{90}{Q4}}};
	\draw (Q4.S) [short, -, l_=VDD] to ++(-1,0) coordinate(VDD);
	\draw (Q4.bulk) [short, -*] to (Q4.D);
	\draw (Q4.D) node[pmos, bulk, rotate=90, anchor=S, xscale=1](Q5){\rotatebox{-90}{Q5}};
	\draw (Q5.bulk) [short, -] to (Q5.S);
	\draw (Q5.D) [short, -] to (Q5.D -| Q1.G) to (Q2.G -| Q1.G);
	\node [circ] at (Q2.G -| Q1.G){};
	\draw (Q5.D -| Q1.G) node[nmos, anchor=S, rotate=-90, xscale=-1](Q6){\scalebox{1}[-1]{\rotatebox{-90}{Q6}}};
	\node [circ] at (Q5.D -| Q1.G){};
	\draw (Q6.D) [short, l=GND, -] to ++(1,0) coordinate(GND);
	\draw (Q4.G) [short, -] to (Q4.G -| Q6.S) to (Q6.S);
	
	\draw (Q3.G) [short, -, l_=clk] to (Q3.G -| VDD);
	\draw (Q3.S) [short, -, l=GND] to ++(-1,0);
	
	\draw (Q5.G) [short, -] to (Q6.G);
	\node [circ] at (Q6.G){};
	\draw (Q6.G) [short,-,l=clk] to (Q6.G -| GND);
	
	%COOOOOORDS
	%\path (temp) \coord(temp);
\end{circuitikz}
	
	\caption{Схема устройства выборки-хранения}
	\label{ct:bootstrapped_switch}
\end{figure}

\begin{figure}[ht]
	\centering
	\input{Dissertation/images/lna_balun.tikz}
	
	\caption{Схема электрическая принципиальная рассматриваемого МШУ}
	\label{ct:lna_balun_wo_bias}
\end{figure}

\begin{figure}[ht]
	\centering
	\includegraphics[width=0.8\linewidth]{lna_gain_nf.pdf}
	
	\caption{Результаты моделирования коэффициент усиления и шума рассматриваемого МШУ}
	\label{ct:lna_gain_nf}
\end{figure}

\begin{figure}[ht]
	\centering
	\includegraphics[width=0.8\linewidth]{lna_s11_s22.pdf}
	
	\caption{Результаты моделирования возвратных потерь по входу и выходу рассматриваемого МШУ}
	\label{ct:lna_s11_s22}
\end{figure}

\begin{figure}[ht]
	\centering
	\includegraphics[width=0.8\linewidth]{phase_diff.pdf}
	
	\caption{Разница фаз между диффиренциальными компонентами на выходе рассматриваемого МШУ согласно результатам моделирования}
	\label{ct:phase_diff}
\end{figure}

\begin{figure}[ht]
	\centering
	\begin{circuitikz}[american, scale=1, transform shape]
		\def\killdepth#1{{\raisebox{0pt}[\height][0pt]{#1}}} \path (0,0) -- (2,0); % bounding box
		\ctikzset{transistors/arrow pos=end}
		\ctikzset{transistors/scale=1}
		
		\draw (0,0) node[left]{\(v_{RF+}\)} [short, o-] to [C, capacitors/scale=0.8] ++(2,0) node[npn, anchor=B](Q1){Q1} coordinate(tmp);
		\draw (Q1.C) [short, -] to ++(0, 4) node[vcc](VCC){\(V_{CC} = 2.5~V\)};
		\draw (Q1.E) [short, -] to [R] ++(0,-2) node[npn, anchor=C](Q2){Q2} (Q2.E) node[ground](GND){} (Q2.B) [short, -] to (Q2.B |- Q2.C) to (Q2.C) node[circ]{};
		\draw (Q1.E) [short, *-] to ++(2,0) node[npn, anchor=B](Q3){Q3};
		\draw (Q3.E) [short, -] to ++(0,-1) to [R] ++(0,-2) node[ground](GND){};
		\draw (Q3.C) [short, -] to ++(0,1) node[circ]{} to ++(-1,0) node[npn, anchor=E](Q4){Q4} to ++(2,0) node[npn, anchor=E, xscale=-1](Q5){\scalebox{-1}[1]{Q5}};
		\draw (Q4.C) [short, -] to ++(0,1) to [R] ++(0,2) node[vcc](VCC){\(V_{CC}\)};
		\draw (Q5.B) node[npn, anchor=B](Q6){Q6};
		\draw (Q6.E) [short, -] to ++(1,0) node[circ]{} coordinate(tmp) to ++(0,-1) node[npn,anchor=C, xscale=-1](Q7){\scalebox{-1}[1]{Q7}};
		\draw (tmp) [short, -] to ++(1, 0) node[npn, anchor=E, xscale=-1](Q8){\scalebox{-1}[1]{Q8}};
		\draw (Q7.E) [short, -] to ++(0,-1) to [R] ++(0,-2) node[ground](GND){};
		\draw (Q3.E) node[circ]{} [short, -] to [R] (Q7.E) node[circ]{};
		\draw (Q3.E) ++(0,-1) coordinate(tmp) node[circ]{} [short, -] to [C] (tmp -| Q7.E) node[circ]{};
		\draw (Q8.C) [short, -] to ++(0,1) to [R] ++(0,2) node[vcc](VCC){\(V_{CC}\)} coordinate(VCC);
		
		\draw (Q5.C) [short, -] to ++(1.2,1.2) coordinate(tmp) to (tmp -| Q8.C) node[circ]{};
		\draw (Q6.C) [short, -] to ++(-1.2, 1.2) coordinate(tmp) to (tmp -| Q4.C) node[circ]{};
		
		\draw (Q5.B) node[circ]{} [short, -] to ++(0,-1) node[ocirc]{} ++(0, -0.3) node(a){\(v_{LO+}\)};
		\draw (Q4.B) [short, -] to ++(0, -1) node[ocirc]{} ++(0, -0.3) node(a){\(v_{LO-}\)};
		\draw (Q8.B) [short, -] to ++(0, -1) node[ocirc]{} ++(0, -0.3) node(a){\(v_{LO-}\)};
		
		\draw (Q7.B) [short, -] to ++(2, 0) node[circ]{} node[npn, anchor=E, xscale=-1](Q9){\scalebox{-1}[1]{Q9}};
		\draw (Q9.E) [short, -] to [R] ++(0,-2) node[npn, anchor=C, xscale=-1](Q10){\scalebox{-1}[1]{Q10}} (Q10.E) node[ground]{} (Q10.B) [short, -] to (Q10.B |- Q10.C) to (Q10.C) node[circ]{};
		\draw (Q9.C) [short, -] to ++(0, 4) node[vcc]{\(V_{CC}\)};
		\draw (Q9.B) [short, -o] to [C, capacitors/scale=0.8] ++(2,0) node[right]{\(v_{RF-}\)};
		
		%COOOOOORDS
		\path (tmp) \coord(tmp);
	\end{circuitikz}
	
	\caption{Схема электрическая принципиальная сверхширокополосного смесителя}
	\label{ct:mixer_1_18}
\end{figure}

\begin{figure}[ht]
	\centering
	\begin{circuitikz}[american, scale=1, transform shape]
		\def\killdepth#1{{\raisebox{0pt}[\height][0pt]{#1}}} \path (0,0) -- (2,0); % bounding box
		
		\ctikzset{bipoles/amp/width=1.2}
		\draw (0,0) to[amp,t=LNA,l_=$F{=}12\,$dB, o-] ++(4,0);
		
		%COOOOOORDS
		%\path (tmp) \coord(tmp);
	\end{circuitikz}
	
	\caption{Структурная схема тракта приемника МИЧ \numrange{1}{1.8} ГГц и \numrange{1.8}{3.3} ГГц}
	\label{ct:struct_1_8_3_3}
\end{figure}

\begin{figure}[ht]
	\centering
	\includegraphics[width=0.7\linewidth]{fig.pdf}
	
	\caption{Оценка частоты тракта \numrange{1}{1.8} ГГц и \numrange{1.8}{3.3} ГГц}
	\label{ct:freq_estimate_1g_3g3}
\end{figure}

\begin{figure}[ht]
	\centering
	\includegraphics[width=0.6\linewidth]{IFM_Struct.pdf}
	
	\caption{Структурная схема частотного дискриминатора диапазона \numrange{1}{1.8} ГГц и \numrange{1.8}{3.3}}
	\label{ct:freq_estimate_1g_1g8}
\end{figure}

\begin{figure}[ht]
	\centering
	\includegraphics[width=1\linewidth]{IFM_Struct2.pdf}
	
	\caption{Структурная схема частотного дискриминатора диапазона \numrange{3.3}{6.3} ГГц}
	\label{ct:freq_estimate_3g3_6g3}
\end{figure}

\begin{figure}[ht]
	\centering
	\includegraphics[width=1\linewidth]{IFM_Struct3.pdf}
	
	\caption{Структурная схема частотного дискриминатора диапазона \numrange{6.3}{11.5} ГГц}
	\label{ct:freq_estimate_6g3_12g}
\end{figure}

\begin{figure}[ht]
	\centering
	\includegraphics[width=1\linewidth]{IFM_Struct4.pdf}
	
	\caption{Структурная схема частотного дискриминатора диапазона \numrange{11.5}{18} ГГц}
	\label{ct:freq_estimate_11g5_18g}
\end{figure}