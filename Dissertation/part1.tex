\chapter{Анализ современных трактов приема сверхширокополосных сигналов}

Современные тракты приемопередающих устройств сверхширокополосных сигналов широко распространены не только в сфере радиоэлектронной борьбы и разведки, но и в гражданском сегменте. Например, любой современный автомобиль снабжается системами мониторинга дорожной обстановки, помощи водителю и контроля состояния водителя, а также пассажиров. Во многих коммуникационных системах возникает необходимость в определении номера занятого канала (быстрого определения несущей частоты) и т.~п.

Все перечисленные секторы применения электронных устройств, работающих с сверхширокополосными сигналами, объединяет важный аспект, область их применения практически всегда связана с безопасностью человека, поэтому критическим параметром для подобных систем всегда остается быстродействие и низкая вероятность пропуска или ложной тревоги.

Интерес к сверхширокополосным системам растет также благодаря надежде на достижение высоких скоростей обмена информацией по сверхширокополосным беспроводным каналам связи, которые к тому же расположены в нелицензируемых диапазонах частот.

В рамках данной главы будут рассмотрены структуры трактов приемников сверхширокополосных сигналов применяемые в радарах, пеленгаторах и пассивных локаторах.

Сверхширокополосные системы передачи данных находят все более широкое применение \cite{wosbib1}, \cite{vakbib1}, \cite{vakbib2}, \cite{scbib1} в области медицины []-[], организации корпоративных и производственных сетей датчиков []-[], а также [],[] и []...

\section{Структуры современных сверхширокополосных приемных трактов}

Для начала следует дать определение сверхширокополосных сигналов как таковых, что принципиально отличает их от узкополосных сигналов. Сверхширокополосным сигналом называется сигнал, обладающий шириной спектра больше 500 МГц (условно для диапазона частот от 3.1 ГГц до 10.6 ГГц) или же спектральные свойства которого удовлетворяет условию:

\begin{equation*}
\frac{f_H - f_L}{(f_H + f_L)/2} \geqslant 0.25 ,
\label{eq:ubw_condition}
\end{equation*}
где \(f_H\) и \(f_L\) -- верхняя и нижняя границы спектра в пределах которых сконцентрировано не менее 90\% энергии сигнала соответственно.

\fixme{Если при рассмотрении сверхширокополосных сигналов не учитывать время накопления или обработки сигналов, то под понятие сверхширокополосных также попадают сигналы с линейной частотной модуляцией, которые по своей природе таковыми не являются и представляют собой узкополосные сигналы (Ширина спектра порядка нескольких десятков мегагерц) линейно перестраиваемые по частоте. Потому можно выделить две группы сверхширокополосных сигналов, методы приема и обработки которых кардинально различаются.}

Проведен аналитический обзор существующих патентов и коммерческих продуктов с целью создания полноценной классификации существующих систем приема сверхширокополосных сигналов, а также методов повышения быстродействия, помехоустойчивости и стабильности \cite{jia_4-bit_2020}, \cite{cheng_introduction_2021, lin_60-ghz_nodate, gray_analysis_2009}, \cite{nagulu_ultra-wideband_2021, rahimpour_design_2019, rucker_013m_2009, pelgrom_matching_1989-1, du_112-gss_2019, hartmann_low-power_2007, saha_6-20_2012, johansen_analysis_2005, shahramian_millimeter-wave_2011, du_256-gss_2018, dyskin_wideband_2016, noauthor_photonic_nodate, noauthor_microwave_2005}.

Предложена классификация способов оценки частоты применяемых в современных приемниках индексации частоты, приведенная на рисунке \ref{ct:classification}.

\begin{figure}[ht]
	\centering
	\includegraphics{Classification.pdf}	
	\caption{Классификация способов оценки частоты сверхкоротких импульсов -- pdf\_tex}
	\label{ct:classification}
\end{figure}

\subsection{Супергетеродинные приемники}

\subsection{Приемники прямого преобразования}

\subsection{Автокорреляционные приемники}

\begin{figure}[ht]
	\centering
	\resizebox{\linewidth}{!}{\input{Dissertation/images/IFM Struct Simple.drawio.pdf_tex}}
	
	\caption{Обобщенная схема простейшего автокорреляционного приемника -- pdf\_tex}
	\label{ct:autocorr_struct_simple}
\end{figure}

\subsection{Приемник измерения мгновенной частоты (IFM)}

Типичная структурная схема приемника измерения (оценки) частоты представлена на рисунке \ref{ct:ifm_struct_simple}. Так как в настоящее время большинство устройств обработки данных принимают данные исключительно в цифровом виде, в данном разделе не будут рассмотренны схемы с аналоговым видеовыходом.

\begin{figure}[ht]
	\centering
	\resizebox{\linewidth}{!}{\input{Dissertation/images/IFM Struct Simple.pdf_tex}}
	
	\caption{Обобщенная схема простейшего приемника измерения мгновенной частоты -- pdf\_tex}
	\label{ct:ifm_struct_simple}
\end{figure}

Измерение частоты в приемнике измерения мгновенной частоты происходит благодаря использованию линий задержки (аналоговых электрических, цифровых, оптических) и частотной зависимости разницы фаз задержанного и незадержанного сигналов. Разница фаз вызванная прохождением сигнала через линию задержки описывается выражением \eqref{eq:phase_diff_ifm}.

\begin{equation}
	\theta = \omega \tau ,
	\label{eq:phase_diff_ifm}
\end{equation}
где \(\omega\) и \(\tau\) - частота сигнала и величина линии задержки соответственно.

Для оценки частоты сигнала согласно выражению \eqref{eq:phase_diff_ifm} с известной величиной линии задержки, необходимо измерить разницу фаз сигналов. Для измерения разницы фаз сигналов удобно на выходе приемника представить сигнал в аналитической форме вида \( s(t) = \cos{\theta} + j \sin{\theta} \). Таким образом, выходные сигналы приемника представляются как

\begin{equation}
	\begin{aligned}
		Q &= A \sin{\theta}, \\
		I &= A \cos{\theta}.
	\end{aligned}
	\label{eq:quadrature_ifm}
\end{equation}

Зная величины квадратурных выходных сигналов приемника \(Q\) и \(I\), используя \eqref{eq:atan2_ifm} определяем значение аргумента обоих сигналов.

\begin{equation}
	\theta = \arctan(Q/I) = \omega \tau
	\label{eq:atan2_ifm}
\end{equation}

В соответствии с \eqref{eq:atan2_ifm} находим оценку частоты сигнала как

\begin{equation}
	\omega = \frac{\theta}{2 \pi \tau}.
	\label{eq:ifm_freq_estimation}
\end{equation}

Зависимость оценки частоты сигнала от частоты входного сигнала представлена на рисунке \ref{ct:ifm_out_signal}. Как можно видеть функция имеет периодический характер и прямо пропорционально зависит от величины задержки \(\tau\) (На рисунке \ref{ct:ifm_out_signal} \(\tau = 400~\textrm{пс}\), что соответствует максимальной индексируемой частоте \(f_{max} = 2.5~\textrm{ГГц}\) ).

\begin{figure}[ht]
	\centering
	\includegraphics[width=0.6\linewidth]{ifm_out_signal.pdf}
	
	\caption{Выходные сигналы частотного дискриминатора во временной и частотной областях}
	\label{ct:ifm_out_signal}
\end{figure}

Основной проблемой возникающей при использовании приемников МИЧ, является работа при множестве сигналов на входе (два и более). Так как при воздействии на частотный дискриминатор более чем одного сигнала, фазовый сдвиг оцениваемый устройством перестает быть показательным и теряет связь со значением несущей частоты сигналов на входе.

Одним из решений проблемы множественного воздействия на вход приемника, в работе \cite{choi2014,gruchalla_instantaneous_2006}, были представлены структуры многоканальных приемников МИЧ. Разделение на каналы позволяло работать с несколькими сигналами одновременно, однако при этом увеличивается сложность структуры, массо-габаритные характеристики, время проектирования, энергопотребление.


ФРАГМЕНТ ДЛЯ ОПИСАНИЯ ДИСКРИМИНАТОРА С КОМПАРАТОРОМ

Выходное напряжение на выходе дискриминатора описывается выражением
\begin{equation*}
	V_{s} = \sin{2 \pi f \tau_{1}}.
\end{equation*}

Так как функция \(\sin{x}\) периодична и переходит через 0 с периодом \(\pi\) в точках \(2 f \tau_{1} \in \mathbb{Z} \).

\subsection{Прочие и гибридные виды приемников}

\section{Принципы и методы оценки быстродействия сверхширокополосных систем}

\subsection{Понятие быстродействия, метрика и т.п.}

\section{Принципы действия и характеристики современных трактов приема сверхширокополосных импульсных сигналов}

\section{Главный тракт приема сверхширокополосных сигналов с точки зрения назначения и применения}
Главный тракт приема в радиочастотных системах выполняет функции преселекции и усиления сигналов в необходимой полосе частот. Из расположения главного тракта в структуре приемной аппаратуры и выполняемых функций становится очевидно его доминирующее влияние на производительность всей системы в целом.

Проблемы проектирования трактов приема сверхширокополосных сигналов в интегральном исполнении активно обсуждаются на протяжении последних Х лет, такими авторами как ...

Наиболее широкое распространение получили приемопередатчики на основе сигналов с прямым расширением спектра, сигналов с псевдослучайной перестройкой по частоте, а также импульсное радио с скачками по времени. Из вышеперечисленных, большинство работ посвящено структурам, работающим с импульсными сигналами, как схемам с наиболее выгодной практической реализацией. Импульсное радио позволяет получить приемлемую скорость передачи данных, возможность определения расстояния и  местоположения объектов в пространстве, а также простую реализацию устройства с малым энергопотреблением [], [] и [].

\subsection{Высокоскоростные сети обмена информацией}

\subsection{Низкоскоростные сети (Сети датчиков, сенсоров или вещей)}

\subsection{Устройства распознавания образов}

\subsection{Устройства определения параметров объектов (Радары)}
Приемники предназначенные для определения параметров принимаемых сигналов значительно отличаются от информационных приемников. Информационные приемники предназначены для восстановления иннформации передаваемой передатчиком. Приемники сигналов радаров не восстанавливают передаваемую информацию, а определяют такие характеристики сигнала, как ширина импульса, форма, частота, частота следования импульсов и угол прихода сигнала (Angle of Arrival).

Исходя из специфики выполняемых задач, подобные приемники должны удовлетворять следующим требованиям:
\begin{itemize}
\item Широкий частотный диапазон;

\item Высокая чувствительность и динамический диапазон;

\item Быстродействие обработки и идентификации характеристик принимаемых сигналов;

\item Возможность обработки двух и более сигналов одновременно;

\item Устойчивость к внешним воздействиям (Широкий температурный диапазон, высокая входная мощность, стойкость к ионизирующему излучению).
\end{itemize}

 Следует также подчеркнуть причину по которой подходы к проектированию связных сверхширокополосных приемников и приемников обнаружения сверхкоротких импульсов (а также оценки их характеристик) так сильно различаются. При проектировании связного приемника и в целом приемо-передающего тракта, параметры и характеристики передаваемых сигналов заранее определены, известны спектральные свойства сигналов несущих информацию.
 В то же время, проектирование приемника обнаружения сверхкоротких импульсов не может осуществляться по тем же методам, что и связные, так как принимаемые сигналы заведомо неизвестны, отсутствует какая-либо синхронизация на приеме и передаче и самое важное, отсутствует цель как таковая прием информационных сообщений. Зондирующие импульсы радарных систем не несут в себе какой-либо информации, информацией для принимающей стороны является сам факт детектирования сигналов данного вида, определения их принадлежности, направления излучения.

\section{Виды сверхширокополосных сигналов}

\subsection{Линейная частотная модуляция (ЛЧМ или FM Chirp)}

\subsection{Сигнал с псевдослучайной перестройкой частоты (FHHS)}

\subsection{Сигналы с двоичной фазовой манипуляцией (BPSK)}

\subsection{Сигналы с полифазным кодированием, расширяющими кодами (DSSS)}

\subsection{Сверхкороткие импульсы (Гауссовские импульсы)}

\section{Факторы ухудшающие быстродействие обработки сверхширокополосных сигналов в реализации тракта приема системой на кристалле}

\section{Источники возникновения ошибок и расхождений фазы сигналов, подходы к уменьшению их влияния}

\section{Выводы по разделу}