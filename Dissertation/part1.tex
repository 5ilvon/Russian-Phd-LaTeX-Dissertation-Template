\chapter{Анализ современных трактов приема сверхширокополосных сигналов}

Современные тракты приемо-передающих устройств сверхширокополосных сигналов широко распространены не только в сфере радиоэлектронной борьбы и разведки, но и в гражданском сегменте. Например, любой современный автомобиль снабжается системами мониторинга дорожной обстановки, помощи водителю и контроля состояния водителя, а также пассажиров. Во многих коммуникационных системах возникает необходимость в определении номера занятого канала (быстрого определения несущей частоты) и т.~п.

Все перечисленные секторы применения электронных устройств, работающих с сверхширокополосными сигналами, объединяет важный аспект, область их применения практически всегда связана с безопасностью человека, поэтому критическим параметром для подобных систем всегда остается быстродействие и низкая вероятность пропуска или ложной тревоги.

Интерес к сверхширокополосным системам растет также благодаря надежде на достижение высоких скоростей обмена информацией по сверхширокополосным беспроводным каналам связи, которые к тому же расположены в нелецензируемых диапазонах частот.

В рамках данной главы будут рассмотрены структуры трактов приемников сверхширокополосных сигналов применяемые в радарах, пеленгаторах и пассивных локаторах.

Сверхширокополосные системы передачи данных находят все более широкое применение \cite{wosbib1}, \cite{vakbib1}, \cite{vakbib2}, \cite{scbib1} в области медицины []-[], организации корпоративных и производственных сетей датчиков []-[], а также [],[] и []...

Для начала следует дать определение сверхширокополосных сигналов как таковых, что принципиально отличает их от узкополосных сигналов. Сверхширокополосным сигналов называется сигнал, обладающий шириной спектра больше 500 МГц (условно для диапазона частот от 3 ГГц до 10 ГГц) или же спектральные свойства которого удовлетворяет условию:

\[ \frac{f_H - f_L}{(f_H + f_L)/2} \geqslant 0.2 , \]

где \(f_H\) и \(f_L\) - верхняя и нижняя границы спектра в пределах которых сконцентрировано не менее 90\% энергии сигнала соответственно.

Если при рассмотрении сверхширокополосных сигналов не учитывать время накопления или обработки сигналов, то под понятие сверхширокополосных также попадают сигналы с линейной частотной модуляцией, которые по своей природе таковыми не являются и представляют собой узкополосные сигналы (Ширина спектра порядка нескольких десятков мегагерц) линейно перестраиваемые по частоте. Потому можно выделить две группы сверхширокополосных сигналов, методы приема и обработки которых кардинально различаются.

Проведен аналитический обзор существующих патентов и коммерческих продуктов с целью создания полноценной классификации существующих систем приема сверхширокополосных сигналов, а также методов повышения быстродействия, помехоустойчивости и стабильности \cite{jia_4-bit_2020}, \cite{cheng_introduction_2021, lin_60-ghz_nodate, gray_analysis_2009}, \cite{nagulu_ultra-wideband_2021, rahimpour_design_2019, rucker_013m_2009, pelgrom_matching_1989-1, du_112-gss_2019, hartmann_low-power_2007, saha_6-20_2012, johansen_analysis_2005, shahramian_millimeter-wave_2011, du_256-gss_2018, dyskin_wideband_2016, noauthor_photonic_nodate, noauthor_microwave_2005}.

\section{Принципы и методы оценки быстродействия сверхширокополосных систем}

\subsection{Понятие быстродействия, метрика и т.п.}

\section{Главный тракт приема сверхширокополосных сигналов с точки зрения области применения и назначения}
Главный тракт приема в радиочастотных системах выполняет функции преселекции и усиления сигналов в необходимой полосе частот. Из расположения главного тракта в структуре приемной аппаратуры и выполнямых функций становится очевидно его доминирующее влияние на производительность всей системы в целом.

\subsection{Высокоскоростные сети обмена информацией}

\subsection{Низкоскоростные сети (Сети датчиков, сенсоров или вещей)}

\subsection{Устройства распознования образов}
вап вап ав

\subsection{Устройства определения параметров объектов (Радары)}

\section{Виды сверхширокополосных сигналов}

\subsection{Линейная частотная модуляция (ЛЧМ или FM Chirp)}

\subsection{Сигнал с псевдослучайной перестройкой частоты (FHHS)}

\subsection{Сигналы с двоичной фазовой манипуляцией}
выаывпварувапыфав вып вап 

\subsection{Сигналы с полифазным кодированием, расширяющими кодами}

\subsection{Сверхкороткие импульсы (Гауссовские импульсы)}
вапвап ва п

\section{Структуры современных сверхширокополосных приемных трактов}

\subsection{Супергетеродинные приемники}

\subsection{Приемники прямого преобразования}
ываываыаыфавыва

\subsection{Автокорелляционные приемники}

\subsection{Прочие и гибридные виды приемников}

\section{Факторы ухудшающие быстродействие обработки сверхширокополосных сигналов в реализации тракта приема системой на кристалле}
абоба

\section{Источники возникновения ошибок и расхождений фазы сигналов, подходы к уменьшению их влияния}

\section{Выводы по разделу}
вапва пва