\chapter{Анализ современных трактов приема сверхширокополосных сигналов}

Современные тракты приемо-передающих устройств сверхширокополосных сигналов широко распространены не только в сфере радиоэлектронной борьбы и разведки, но и в гражданском сегменте. Например, любой современный автомобиль снабжается системами мониторинга дорожной обстановки, помощи водителю и контроля состояния водителя, а также пассажиров. Во многих коммуникационных системах возникает необходимость в определении номера занятого канала (быстрого определения несущей частоты) и т.~п.

Все перечисленные секторы применения электронных устройств, работающих с сверхширокополосными сигналами, объединяет важный аспект, область их применения практически всегда связана с безопасностью человека. Поэтому критическим параметром для подобных систем всегда остается быстродействие и низкая вероятность пропуска или ложной тревоги.

В рамках данной главы будут рассмотрены структуры трактов приемников сверхширокополосных сигналов применяемые в радарах, пеленгаторах и пассивных локаторах.

Проведен аналитический обзор существующих патентов и коммерческих продуктов с целью создания полноценной классификации существующих систем приема сверхширокополосных сигналов, а также методов повышения быстродействия, помехоустойчивости и стабильности.

\section{Главный тракт приема сверхширокополосных сигналов}

\subsection{Супергетеродинные приемники}

\subsection{Приемники прямого преобразования}

\subsection{Интерферометрические приемники}

\subsection{Многоканальные приемники (мультиплексированные)}

\subsection{Прочие и гибридные виды приемников}

\section{Факторы ухудшающие быстродействие обработки сверхширокополосных сигналов}

\section{Источники возникновения ошибок и расхождений фазы сигналов, подходы к уменьшению их влияния}